\documentclass[10pt]{article}
\usepackage{geometry}

\title {Dodgeball Tactics}
\author {Zach Stecher}

\setlength{\voffset}{-.095in}
\setlength{\headsep}{10pt}

\begin{document}
\maketitle

\section*{Game Overview}
Dodgeball Tactics is a 2-player game to be played head-to-head. The objective is to maneuver your team around the grid-based board and eliminate every player on the opposing team before they eliminate yours. Each unit may move a certain number of spaces per turn. You eliminate enemy players by either throwing a dodgeball at them and landing a hit, or catching a dodgeball thrown at one of your players.

\section*{Game Setup}
\begin{itemize}
\item You will need: At least 3 6-sided dice, tokens or some other way to keep track of dodgeballs in each player's pool, the game board, the unit tokens.
\item Each team starts with 4 Assault units, 4 Defender units, and 1 Ace unit.
\item Before the game starts, players arrange their units along the back row however they see fit, with the first player doing so first. Each space may only hold 1 unit. This is where these units will start the game.
\item Each player begins with 3 dodgeballs in their pool, for a total of 6. Use beads or dice to keep track of this.
\end{itemize}

\section*{Turn Structure}
Before the game begins, roll dice to determine which player will go first.

\begin{enumerate}
\item The first player may move each of their units once per unit, up to the maximum amount of spaces that unit type is allowed to move. Units may NOT cross the middle line on the board between the two sides. Units may move through other units but may not occupy the same space at the end of their movement.
\item After the first player has decided their unit movements for the turn, the second player moves their units for the turn.
\item Complete combat for the turn (see the Combat section).
\item Each player gives their opponent a ball for each attack they initiated during combat. If the player with more balls did not attack at all, they must give their opponent 1 ball. This is to prevent one person from stockpiling balls and stalling the game.
\item Proceed to the next turn. Whichever player moved second last turn moves first.
	
\end{enumerate}

\section*{Combat}
\begin{itemize}
\item Before combat begins, both players roll 2 dice to see who will attack first. Players alternate selecting their attacks until either no available balls are left or both players decide to stop attacking.
\item To attack, the attacking player selects their unit to attack with, and then selects an opponent's unit as the target. Attacking costs one 'ball' from your pool.
\item The attacking player rolls 3 dice, and then adds or subtracts any applicable bonuses or penalties(See section on bonuses and penalties)
	\begin{itemize}
	\item If the attacking unit has a friendly unit in an adjacent space, the attacking player may choose to 'combine' their attacks by using up the adjacent unit's attack for the turn (and ball from the pool) to add another die to their attack roll. You may only combine once per attack.
	\end{itemize}
\item The defending player then also rolls 3 dice, also adding or subtracting any applicable bonuses or penalties.
\item If the attacking player's total is higher, the target is out of the game.
\item If the defending player's total is higher or the rolls are equal, they dodge or block the attack and are safe. If their total is DOUBLE the attacker's total or more, they catch the ball and the attacker is out. The defender may then also place a previously eliminated unit back into play on any space in the rear-most row on their side. This unit may NOT be the "Ace" unit. If the 'caught' ball is from a combined attack, the defending player chooses which of the combined units is out. Only one attacking unit is eliminated in this situation.
\item Each unit on a team may attack once per turn, but a team may not attack more times than they have balls available.
	\begin{itemize}
	\item Example: If you have 4 balls in your pool and 6 units left on your team, only 4 units may attack this turn, and none of them may attack more than once.
	\end{itemize}
\item Keep track of the amount of balls thrown. These are to be given to your opponent at the end of the turn.
\end{itemize}
\newpage

\section*{Unit Types}
\begin{itemize}
\item Assault: Assault units get +2 to attack rolls and -2 to defense rolls. May move up to 4 spaces.
\item Defender: Defender units get +2 to defense rolls and -2 to attack rolls. May move up to 3 spaces.
\item The Ace: The Ace gets +2 to ALL rolls. The Ace may not be brought back into the game by catching a ball if eliminated. May move up to 5 spaces.
\end{itemize}

\section*{Bonuses and Penalties}
\subsection*{Bonuses}
	\begin{itemize}
	\item Assault units and The Ace get +2 to attack rolls.
	\item Defender units get +2 to defense rolls.
	\item if a unit has a friendly Defender unit in an adjacent space, it gets +1 to defense rolls per adjacent Defender unit.
	\end{itemize}

\subsection*{Penalties}
	\begin{itemize}
	\item Attack rolls get a minus equal to the number of spaces between the attacking unit and the defending unit, divided by 2 rounding down. Determine distance using the straightest possible line and only count full spaces.
		\begin{itemize}
		\item Example: If a unit is 10 spaces away, the attacking roll gets -5. If the unit is 9 spaces away, the attacking role gets -4.
		\end{itemize}
	\item Defender units get -2 to attacking rolls.
	\item Attack rolls get -1 for each unit in between the attacking unit and the target.
	\end{itemize}

\end{document}